\section{Bevezetés}
Úgy gondolom, hogy a szoftverteszteléssel foglalkozó szakemberek szerepe, feladatai eltérhetnek a különböző cégeken, de még a projekteken belül is.
Szerintem ennek egyik fő oka a szoftvertesztelés költségessége. A szakdolgozatomban a szoftvertesztelők szerepét kutatom a szoftvertesztelési módszerek és projektre szánt források közötti kapcsolat függvényében.

A szoftvert tesztelők által elvégzett feladatok a projekttel kapcsolatos elvárásokkal együtt változik.
A tesztelés mértékét és módszereit befolyásolja a kockázat és a projektre szánt források mértéke. Az idő és pénz szűkössége miatt, a meg kell választani, hogy hova tegyük a hangsúlyt a tesztelés során a kockázatok függvényében. \footcite{istqbfoundations}

Maga a minőségbiztosítás egy elég komplex, szerteágazó folyamat, aminek a tesztelés csak egy része\footcite{softwarequalityassurance2016}, ezért én ezen belül is csak az agilis környezetben történő tesztelést fogom vizsgálni.