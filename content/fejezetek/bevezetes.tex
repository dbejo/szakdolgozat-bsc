\section{Bevezetés}
Úgy gondolom, hogy a szoftverteszteléssel foglalkozó szakemberek szerepe, feladatai eltérhetnek a különböző cégeken, de még a projekteken belül is.
Szerintem ennek egyik fő oka a szoftvertesztelés költségessége. A szakdolgozatomban a szoftvertesztelési módszerek és projektre szánt források közötti kapcsolatot kutatom.

A szoftvert tesztelők által elvégzett feladatok a projekttel kapcsolatos elvárásokkal együtt változik.
A tesztelés mértékét és módszereit befolyásolja a céges kúltúra, szabványoknak történő megfelelés, ügyfél kívánsága és az erre szánt források.

Maga a minőségbiztosítás egy elég komplex, szerteágazó folyamat, aminek a tesztelés csak egy része\footcite{softwarequalityassurance2016}, ezért én ezen belül is csak az agilis környezetben történő tesztelést fogom vizsgálni.