\section{Szoftvertesztelők szerepe a fejlesztési folyamatban}
\subsection{Mi a szoftvertesztelők feladata?}
\subsection{Hogyan illeszkednek a szoftvertesztelők a szoftverfejlesztési életciklusba?}
Az emberi hibák a szoftverfejlesztési életciklusban bármikor bekövetkezhetnek.
Ezeknek a következményei lehetnek triviálisak, vagy akár katasztrófálisak is, ezért fontos, hogy a tesztelés folyamatosan történjen a fejlesztés és az azt követő fázisokban is, hogy felfedezzük és csökkentsük a használat során észlelhető hibák számát.
A szoftvertesztelés során nemcsak a rendszer funkcionalitását ellenőrizzük, hanem feladatunk a gyengepontok keresése is. Ez magában foglalja például annak vizsgálatát, hogy a felhasználó véletlenül hibás adatokat táplál-e be a rendszerbe, és hogy a rendszer hogyan kezeli ezeket a hibás adatokat.
Emellett a tesztelés kiterjedhet olyan szituációk elemzésére is, amikor szándékos, kártékony támadások érik a rendszert, és ezek hatásait is felmérjük. Célunk, hogy biztosítsuk a rendszer megbízhatóságát és biztonságát a véletlen hibáktól és a potenciális támadásoktól egyaránt. \cite[9]{istqbfoundations}
\subsection{A szoftvertesztelők és más csapattagok (fejlesztők, üzleti elemzők, termékmenedzserek) közötti együttműködés}